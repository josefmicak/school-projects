\chapter{Úvod}
\label{sec:Introduction}
Tato bakalářská práce se zabývá deskovou hrou \emph{Čínská dáma}. Ačkoli by tomu název mohl napovídat, tato hra toho nemá příliš společného s mnohem známější hrou, kterou známe pod názvem Dáma. Navíc existuje více interpretací pravidel, týkajících se počtu herních polí a kamenů hráče, nebo samotných tahů. Nejprve je tedy čtenář seznámen s důležitými teoretickými podklady týkajícími se hry –- původem hry, principem hry a zvoleným souborem pravidel. Následuje objektová analýza hry, ze které se vychází při samotné implementaci.

Cílem práce je lidskému hráči umožnit odehrání hry proti jednomu až pěti počítačovým hráčům nastavitelných obtížností. Práce zahrnuje popis jednotlivých obtížností počítačového hráče a~v~samotné hře je implementován simulátor, ve kterém si lidský hráč může nasimulovat hru mezi volitelným počtem počítačových hráčů volitelných obtížností a sledovat průběh hry a její výsledek. Pro praktickou ukázku chování nepřátel podle jejich obtížností práce obsahuje analýzu většího množství her mezi počítačovými hráči stejných i různých obtížností.
\endinput