\chapter{Závěr}
Cílem této bakalářské práce bylo popsat pravidla a principy deskové hry Čínská dáma a následně provést implementaci umožňující uživateli hru proti jednomu nebo více počítačovým hráčům. 

Všechny cíle bakalářské práce byly splněny, první část práce podrobně popisuje základní principy této deskové hry a také poukazuje na značnou nejednotnost co se týče pravidel hry, přičemž vysvětluje, s využitím kterých konkrétních pravidel je počítáno v rámci této práce. Další části práce poté mimo jiné vysvětlují zvolené postupy při implementaci jednotlivých částí a prvků hry. Uživatel si může při vytváření hry kromě počtu počítačových hráčů, proti kterým bude hrát, zvolit také jejich obtížnost. Na výběr je z celkem tří různých obtížností, dovednosti jednotlivých počítačových hráčů podle jejich obtížnosti jsou pak prakticky porovnávány v poslední části, ve které je na základě více kritérií zanalyzováno větší množství her mezi počítačovými hráči stejných i různých obtížností.

Mezi možná vylepšení nebo rozšíření vytvořené aplikace bychom mohli zařadit možnost začínat v jiném trojúhelníku na herní desce nebo vlastní výběr barev jednotlivých hráčů, v této práci jsou barvy neměnné zejména z důvodu usnadnění odlišitelnosti různých hráčů. Dále by uživatel mohl mít možnost vybrat si vlastní interpretaci pravidel, někteří uživatele by totiž mohli preferovat ovládání více sad kamenů najednou nebo povolení tahů do všech polí na herní desce. V neposlední řadě také existuje prostor pro vylepšení dovedností počítačového hráče s nejvyšší obtížností, ačkoli by v~takovém případě vzniklo riziko, že u her s více počítačovými hráči s touto obtížností by pak výpočet jejich tahů trval příliš dlouhou dobu.

\hfill Josef Micak
\endinput